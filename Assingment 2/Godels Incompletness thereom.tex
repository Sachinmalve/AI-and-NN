  \documentclass{article}
  
\begin{document}

\title{\vspace{-5cm}Gödel's incompleteness theorems}
\author{Sachin Malve \\
	19111049 \\
 	5th Semester \\ 
	Biomedical Engineering\\
	}
\maketitle 
 \hrulefill
\section{Introduction}
Gödel's theorem is in the field of Mathematical logic and he gave 2 theorems. First theorem is simple terms said There are certain problems in Mathematics which cannot be solved using certain set of Axioms. And the second theorem said You cannot prove that a system of axioms is consistent, unless you use a different set of axioms.
He said that every formal system is either be incomplete or inconsistent.\\
\textbf{What is Completeness and Consistency ?} \\
In mathematical logic, a theory is complete if, for every closed formula in the theory's language, that formula or its negation is demonstrable. And A consistent theory is one that does not lead to a logical contradiction.
\section{Gödel's incompleteness theorems}
\subsection{First Incompleteness Theorem  } 
Any Consistent Formal system F within which a certain amount of elementary arithmetic can be carried out is incomplete i.e. there are statements of the language of F which can neither be proved nor disproved in F
\subsection{Second Incompleteness Theorem  } 
For each formal system F containing basic arithmetic, it is possible to canonically define a formula Cons(F) expressing the consistency of F. This formula expresses the property that "there does not exist a natural number coding a formal derivation within the system F whose conclusion is a syntactic contradiction." The syntactic contradiction is often taken to be "0=1", in which case Cons(F) states "there is no natural number that codes a derivation of '0=1' from the axioms of F."   
\section{Why Gödel's Theorem are Important ?}
His theorem shows that there are limitations within every system and there are things which cannot be proved even if they are true. It also opens up the argument that no theory in physics, math or any vertical can ever be 100% certain
His theorem allowed for derivation of certain limitations in computations such as Halting problem and His theorem is even used to logically prove about existence of God.
His theorem showed the limitations of logical theorems and were foundation of computer science 

\end{document}