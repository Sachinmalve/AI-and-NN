 \documentclass{article}
  
\begin{document}

\title{\vspace{-5cm}Moravec's paradox}
\author{Sachin Malve \\
	19111049 \\
 	5th Semester \\ 
	Biomedical Engineering\\
	}
\maketitle 
 \hrulefill
 \\
 \\
 
\section{Introduction}
 Moravec's paradox states that contrary to the common belief Reasoning is easy but sensorimotor skills are hard for machines to process. The things we are least conscious are harder to reverse engineer. \\
  
\section{The biological basis of human skills}
Moravec said they might be harder to replicate because natural selection had much more time perfecting there design however abstract thinking and reasoning are relatively newer processes.The oldest human skills are largely unconscious and so appear to us to be effortless.\\
Examples of such skills are Moving,walking, Holding something in hand recognizing faces, attention, visualization and anything to do with perception. And example of reasoning skills would be Mathematics,, logic and scientific reasoning. \\

\section{Historical influence on artificial intelligence}
AI researchers thought that they would be creating thinking machines in just a few decades because they thought they would replicate reasoning cause it's harder for human minds. But they were wrong it was easier to solve harder problems and hard to solve easy problems like commonsense and vision.\\

\section{Conclusion} 
 Moravec's paradox proves that even we have found solution of harder problems it doesn't mean AI is around the corner, The AI will not able replace at certain tasks.Contrary to common belief Not all human jobs are replaceable.
 
 
 
 
 
 
 
 
 
 
 
 
 \end{document}